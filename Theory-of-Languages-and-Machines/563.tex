\documentclass[10pt,a4paper]{book}
\usepackage{graphicx}

\begin{document}
\small

\begin{flushright}
  \textsf{\textbf{Computational Complexity $|$ 563}}
\end{flushright}

The following three examples in (d), (e), and (f) are not in CNF

\quad

\quad d) $\neg (X1 \vee \neg X2) [NOT problem]$

\quad e) $\neg X2 \vee (X1 \wedge X3) (AND problem)$

\quad f ) $X1 \wedge (X2 \vee (X1 \wedge X3)) (OR problem)$

\quad

These three formulas can be converted to equivalent CNF as in (g), (h), and (i).

\quad

\quad g) $\neg X1 \wedge X2$

\quad h) $(\neg X2 \vee X1) \wedge (\neg X2 \vee X3)$

\quad i) $X1 \wedge (X2 \vee X1) \wedge (X2 \vee X3)$

\quad

CNF is used in SAT problems for detecting conflict and to remember partial

assignments that do not work.

\quad(In the context-free grammar chapter, we learnt about the Chomsky normal

form. Do not mess this CNF with the conjunctive normal form.)

\quad

\textsf{\textbf{2 SAT:}} A SAT problem is called 2 SAT if the number of literals in each disjunction 

(clause) is exactly 2.

\textsf{\textbf{3 SAT:}} A SAT problem is called 3 SAT if the number of literals in each disjunction

(clause) is exactly 3.

\quad SAT is in NP because any assignment of Boolean values to Boolean variables 

that are claimed to satisfy the given logical expression can be verified in polynomial
 
time by a deterministic Turing machine. (Can be proved)

\;

\textsl{\textbf{Prove that 2 SAT problem is in P}}

\textsl{\textbf{Proof:}} Let $\phi$ be an instance of 2 SAT. Generate a graph $G(\phi)$ by the following 

rules.

1. Vertices of the graph G are the variables and their negations. If the number of 

\quad variables is n, then the number of nodes is 2n.

2. If there is a clause $(\neg \alpha \vee \beta)$ or $(\beta \vee \neg\alpha)$ in $\phi$, then an edge $\alpha \rightarrow \beta$ is added in 

G.

3. If there is an edge from $\alpha \rightarrow \beta$ in G, then an edge $\neg\beta \rightarrow \neg\alpha$ is added in 

G.

4. If there is a clause $(\alpha \vee \beta)$ in $\phi$, then two edges $\neg\alpha \rightarrow \beta$ and $\neg\beta \rightarrow \alpha$ are added

in G.

\quad

Let us take an example $\phi = (x_{1} \vee x_{2}) \wedge (x_{1} \vee \neg x_{3}) \wedge (\neg x_{1} \vee x_{2}) \wedge (x_{2} \vee x_{3})$.

1. There are three variables $x_{1}, x_{2}, and x_{3}$ in $\phi$. Thus, the graph contains 6 nodes
 
namely $x_{1}, \neg x_{1}, x_{2},\neg x_{2}, x_{3}, and \neg x_{3}$.

2. $(x_{1} \vee \neg x_{3})$ is in the form $(\beta \vee \neg \alpha)$. Thus, an edge $x_{3} \rightarrow x_{1}$ is added to G.
 
According to the rule (c),an edge $(\neg x_{1} \rightarrow \neg x_{3})$ is also added to G.

3. By the same rule, $x_{1} \rightarrow x_{2}$ and $(\neg x_{2} \rightarrow \neg x_{1})$ are also added to G.

4. $(x_{1} \vee x_{2})$ is in the form $(\alpha \vee \beta)$. Thus, two edges $(\neg x_{1} \rightarrow x_{2})$ and $(\neg x_{3} \rightarrow x_{1})$ 

are added to the graph.

5. By the same rule, $(\neg x_{2} \rightarrow x_{3})$ and $(\neg x_{3} \rightarrow x_{2})$ are added to the graph for

$(x_{2} \vee x_{3})$.

$\phi$ is unsatisfi able if and only if there is a variable x such that there are edges from x to $\neg x$ and $\neg x$ to x in G.

\footnotesize
Suppose such x exists. x may have any value between 0 and 1. Consider x = 1 and an edge x to $\neg x$.
(Consider x as $\alpha$ and $\neg x$ as $\beta$.) It means there is a clause $(\neg x \vee \neg x)$ or $(\neg x \vee \neg x)$ (both are the same)in $\phi$. The clause $(\neg x \vee \neg x)$ returns 0 if x = 1. This makes the total $\phi$ as 0 [because the sub-clauses are attached with $\wedge$] which means unsatisfiable.

\end{document}
