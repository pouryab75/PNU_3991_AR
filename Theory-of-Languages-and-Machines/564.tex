\documentclass[10pt,a4paper]{book}
\usepackage{graphicx}
  % Requires \usepackage{graphicx}

\begin{document}
\small

\begin{flushleft}
  \textsf{\textbf{564 $|$ Introduction to Automata Theory, Formal Languages and Computation}}
\end{flushleft}

\begin{figure}[h]
  \centering
  % Requires \usepackage{graphicx}
  \includegraphics[width=10cm]{1.1}\\
\end{figure}

The same thing is true for an edge $\neg x$ to x in G.

For proving the ‘only if’ part, we have to take the help of contradiction. Consider

$\phi$ is unsatisfi able.

1. If there is a path from $\alpha$ to $\neg\alpha$ for a node $\alpha$, then $\alpha$ must be assigned to false.

2. If there is no path from $\alpha$ to $\neg\alpha$ for a node $\alpha$, then those nodes which are

reachable from $\alpha$ must be assigned to false. [According to the rule that if there is

an edge $\alpha \rightarrow \beta$, then there are clauses $(\neg\alpha \vee \beta)$ or $(\beta \vee \neg\alpha) in \phi.]$

3. Repeat this for all the nodes.

\quad

Now consider the previous graph. There is no edge from $x_{3}$ to $\neg x_{3}$.

If $x_{3}$ is true, then $x_{1}$ must be false, because this was added for the sub-clause $(x_{1} \vee \neg x_{3})$ in $\phi$. There is an edge $x_{1} \rightarrow x_{2}$ for the clause $(¬x_{1} \vee x_{2})$ in $\phi $. If $x_{1}$ is true, then $x_{2}$ is false, which means $x_{3}$ must be
false and $\neg x_{3}$ must be true.

There is an edge $(\neg x_{2} \rightarrow x_{3})$ for the clause $(x_{2} \vee x_{3})$ in $\phi$. If $x_{2}$ is false, then $x_{3}$

must be false.

Here, we are getting the contradiction for $x_{3}$ and $\neg x_{3}$.

It proves that step (2) cannot exist if $\phi$ is unsatisfiable, which justifies that $\phi$ is unsatisfiable if and only if there is a variable x such that there are edges from x to $\neg x$ and $\neg x$ to x in G.

For a 2 SAT problem, if there are n variables, then the existence of an edge from x to $\neg x$ and $\neg x$ to x can be found within 2n steps. This signifies that 2 SAT is in P.

\begin{flushleft}
\emph{\textsf{\textbf{Prove that 3 SAT problem is in NP}}}

\textsf{\textbf{\emph{Proof :}}} A SAT problem $\phi$ is called 3 SAT if the number of literals in each clause is exactly 3. Let F be a CNF SAT problem. We have to convert F to a 3 CNF SAT problem $F'$ such that if F is satisfiable, then $F'$ is also satisfiable. Let the problem F contain clauses $C_{1}, C_{2}….C_{k}$. Here, three cases of C may occur.

\footnotesize
1. Ci contains exactly 3 literals.

2. Ci contains less than 3 literals.

3. Ci contains more than 3 literals.

\: We have to concentrate on cases (2) and (3).

\textsl{\textbf{Solution for Case (2):}} If Ci contains less than 3 literals, then the number of literals may be 2 or 1. If it contains 1 literal, say $L_{1}$, then replace this by $(L_{1} \vee L_{1} \vee L_{1})$.
\end{flushleft}

\end{document}
