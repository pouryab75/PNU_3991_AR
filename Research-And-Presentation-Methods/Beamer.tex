\documentclass [12pt]{beamer}
\usepackage{xcolor}	
\usepackage{tikz}
\usetheme{AnnArbor}
\usetheme{Warsaw}
\usepackage{ragged2e}
\begin{document}

\section*{kholase safahat 22...24}
\subsection*{poriya bedaghi}	
\begin{frame}
\justifying		
if you are interested in more public information then the most effective search tool will be a hierarchical subject guide . in hierarchy or subject directories , you pass through the hierarchy as long as you reach the place you are looking for . in simple operating conditions , subject directories are organized around existing directories , created and filled by humans. search engines use every word in sindh and all the comprehensive labels in the introduction of the page to create very large databases . these databases are automatically generated and maintained by software robots commonly referred to as bots or just robots . under simple operating conditions , search engines work with robots that make nearly every document available on the public web . the contents are then searched very large and fast . finally , some search engines use a technique that actually uses hundreds of visited hundreds of visited , which place the " consumer choice " above the list with any subsequent search .
\end{frame}

\begin{frame}

some search engines list every word per page, while others are just part of the document. the main difference between search engines and subject directories remains. subject directories tend to wait for web pages to be presented by the author, then evaluated and placed in the category of "appropriate hierarchical" category .a search engine, instead of a thematic guide, is usually used to find specific information on the internet. by placing words in the quotation mark , we can further refine the search and instruct the search engine to select only pages that the word online is prior to word research. the suitability ratings for useful information retrieval is critical and increases as the number of web documents increases . most of us don 't have time to determine which web documents are appropriate and useful through millions of possible bumps. obviously , the more the results are , the more we consider the web as a useful information resource.when you don't get the desired results from a search engine,you may have a better chance with a subject guide,because 
	
\end{frame}

\begin{frame}

subject directories are the option of modifying the search based on a specific topic.

	
Overall, the best strategy for retrieving useful and relevant information is to use the same tactics that you would use in a library search. First, begin by analyzing your needs. What are you looking for? If the topic is very specific, search engines such as Google will likely result in hits that are relevant and useful. Alternatively, if you are looking for a broad topic, a subject guide such as Yahoo! would likely return the most successful results. Next, conceptualize your search question and then isolate the keywords in the question by eliminating words that are irrelevant.  You may find it useful to use a special purpose search tool that can most accurately meet your needs. Many search tools are designed with specific aims in mind and, as such, there is no best search tool. Rather, which search tool is best will depend on the kind of search you are doing

\end{frame}
\end{document}	
	