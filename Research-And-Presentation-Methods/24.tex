\documentclass[10pt,a4paper]{book}

\begin{document}

\begin{flushleft}
  \textbf{\!\!\!\!\!\!\!\!\!\!\!\!\!\!\!\!\!\!\!\!24} \qquad CHAPTER TOW
\end{flushleft}

Overall,the best strategy for retrieving useful and relevant information is to use the same tactics that you would use in a library search.First,begin by analyzing your needs.What are you looking for?If the topic is very specific,search engines such as Google will likely result in hits that are relevant and useful.Alternatively,if you are looking for a broad topic,a subject guide such as Yahoo!would likely return the most successful results.Next,conceptualize your search question and then isolate the key-world in the question by eliminating words that are irrelevant.

You may find it useful to use a special purpose search tool that can most accurately meet your needs.Many search tools are designed with specific aims in mind and,as such,thara is not best search tool.Rather,which search tool is best will depend on the kind of search you are doing.The following are a few examples of the different kinds of search tools available with specific aims.

\qquad

\small

$\bullet$ MedNet (http://www.mednets.com/) is devoted exclusively to medical informa

\quad tion.

$\bullet$ Kids search Tools(http://www.rcle.org/ksearch.html)has been developed exclu

\quad  sively for children's searches,to be used by children.


$\bullet$ Search Engine Collosus(http://www.searchenginecolossus.com/)provides search

\quad  engines focused on particular regions and countries.


$\bullet$ If you are not sure which search engine to use,or even what search engines are 

\quad  available,NoodleQuest(http://www.noodletools.com/noodlequest)is a Web site

\quad  with an automated Web form that will generate a list of appropriate search eng

\quad  ines based on both your Internet skills and your search needs.

$\bullet$ Searchability(http://www.Searchability.com/)and NeuvaNet(http://www.noodle

\quad tools . com/noodlequest/) are Web sites that provide listings of specialty search 

\quad engines with advice on how to choose the search engine most appropriate for yo

\quad ur needs . For example,if you are looking for a few good hits fast,NeuvaNet reco

\quad  mmends the use of Google(https://www.google.com/).Vivisimo(https://vivisimo

\quad .com/form?form=Advanced),and Ixquick(https://ixquick.com/).

$\bullet$ If you are looking for a general and broad academic subject and need to focus it,

\quad NeuvaNet recommends search guides such as:

\qquad Encarta Online(http://www.encarta.msn.com/reference)

\qquad Encyclopaedia Britannica(http:///www.britannica.com/)

\qquad Northern Light(http;//www.northernlight.com/search.html)

\qquad Librariand' Index to the Internet(http://lii.org/), or Informine (https://informi

\qquad ne.ucr.edu/)

$\bullet$ If you are looking for biographical information,try using Lives(https://amillionlives

\quad .com/),Biography.com(https://www.biography.com/search/),or Biographical Dici

\quad  tonary(http://www.biography.com/search/),or Biographical Dictonary(http://w

\quad ww.s9.com/biography/).

$\bullet$ There is rapid progress being made in the cataloging and retrieval ( through met

\quad descriptors ) of graphic images that you can use to enhance a research presen 

\quad atag tation.Twoof the larges collections of graphic images and pictures are avail

\quad  able https://ditto.com/and https://www.altsvista.com/sites/search/simage.

$\bullet$ Sound and music files are also difficult to find due to problems in classification.

\quad Moodlogic(www.moodlogic.com)creates search application that allow you to.

\end{document}
