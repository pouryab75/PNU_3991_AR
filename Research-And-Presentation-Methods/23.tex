\documentclass[10pt,a4paper]{book}

\begin{document}

\begin{flushright}
  WHAT IS THE NET? \qquad \textbf{23}
\end{flushright}

!\!\!\!\!\!\!\!\!\!\!be noted that some search engines index every word on every page,while others index only part of the document,such as the title,headings,subheadings,hyperlinks to other sites,and the first twenty lines of text.Further,search engines that have full-text indexing systems claim to pick up every word in the text except commonly occurring stop words such as \emph{a,an,the,is,and,} and \emph{or,} and some search engines discriminate uppercase form lowercase,whereas others store all words without reference to capitalization.

At this point,it should be noted that the distinction between a search engine and a subject guide is becoming blurred,given that most search sites now offer both search options---as they try to provide everything to everyone as one-stop search portals.In spite of this new trend,the basic operational difference between search engines and subject guides remains.Specifically,search engines use robots to search for,and record,as many Web sites as possible.Whereas subject guides tend to wait for Web pages to be submitted by the author,which are then assessed and placed in the appropriate hierarchical subject category.

A search engine,rather than a subject guide,is usually used to find particular information on the Net.However,because search engines index almost every document on the Web,retrieving what you are looking for often results in an onerous activity of selection from hundreds,thousands,or even millions of Web pages that are returned from search queries.Wading through these Web page can consume a tremendous amount of time.To get relevant and useful results,it is important to make use of the search engine's advanced search features including quotation marks and Boolean operators.For example,typing the phrase \emph{online research}into Alta Vista produces about 2.660.000 potential sites with either the word  \emph{research}or the word \emph{online search}to documents in which the words \emph{online}and\emph{research}are both found,by joining the two words with the Boolean operator \emph{and}.For example,typing \emph{online and research}now produces 138.000 successful hits---though,still not few enough to be useful.Fortunately,we can further refine the search by placing the words in quotation marks,thus instructing the search engine to select only pages where the word \emph{online}precedes the word \emph{research}. Now the search for\emph{online research}produces only 62.000 hits but still too many Web sites to be useful.At this point we can continue adding and words to further reduce our search and hone in on pages that are particularly interesting.For example adding the words \emph{and surveys}reduces the number of hits to 643;while the words \emph{online research and focus group and education}produce a manageable 128 hits. 

Relevancy ranling is critical for useful retrieval of information,and becomes more so as the number of Web documents grows,Most of use do not have the time to sort through the possible millions of hits to determine which web documents are relevant and useful.Obviously,the more relevant and useful the results are,the more we are likely to value the search enginc and,in turn,regard the Web as a useful information resource.In particular,when you are not getting the desired results from a search engine,you may have better luck using a subject guide,as subject guide provide the option to refine your search based on a particular topic.For example,searching for e-research within the field of higher education,a subject guide returns only pages about e-research in higher education,not e-research in elementary,middle school,business,or any other field.Searching within a hierarchical category of interest allows you to quickly narrow in on only the most relevant pages . 

\end{document}
